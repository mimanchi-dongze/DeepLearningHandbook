\chapter{自动求导与计算图}
\label{ch:autograd}

PyTorch 的核心引擎是 Autograd,它实现了自动化的反向传播。

\section{计算图与链式法则}
深度学习模型可以看作是有向无环图(DAG),其中节点是张量,边是数学运算。

\section{Vector-Jacobian Product (VJP)}
在反向传播中,我们需要计算输出对输入的导数。对于函数 $\mathbf{y} = f(\mathbf{x})$,其雅可比矩阵 (Jacobian) $J$ 定义为 $J_{ij} = \frac{\partial y_i}{\partial x_j}$。

然而,PyTorch 并不显式计算整个矩阵(因为对于大型参数,这会导致内存溢出),而是计算 **Vector-Jacobian Product**。

\begin{rosetta}{VJP 公式}
    设上一层传回的梯度向量为 $\mathbf{v} = \frac{\partial L}{\partial \mathbf{y}}$,则当前层的梯度计算为:
    \begin{equation}
        \mathbf{v}^T J = \left[ \frac{\partial L}{\partial y_1}, \dots, \frac{\partial L}{\partial y_m} \right] 
        \begin{bmatrix}
            \frac{\partial y_1}{\partial x_1} & \dots & \frac{\partial y_1}{\partial x_n} \\
            \vdots & \ddots & \vdots \\
            \frac{\partial y_m}{\partial x_1} & \dots & \frac{\partial y_m}{\partial x_n}
        \end{bmatrix}
    \end{equation}
    结果即为 $\frac{\partial L}{\partial \mathbf{x}}$。
\end{rosetta}

\section{PyTorch 操作}
\begin{itemize}
    \item \pyfunc{requires\_grad=True}: 标记该张量需要追踪操作以计算梯度。
    \item \pyfunc{backward()}: 触发从当前节点开始的反向传播。
    \item \pyfunc{grad}: 存储计算得到的梯度值。
\end{itemize}
