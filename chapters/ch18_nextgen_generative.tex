\chapter{下一代生成法则}
\label{ch:nextgen_generative}

超越经典的 VAE 与基础扩散模型,当代生成大模型(图像、视频)采用了更强大的引导机制与流模型数学框架。

\begin{notation}
\begin{itemize}
    \item $x$: Input tensor
    \item $\theta$: Model parameters
\end{itemize}
\end{notation}

\mnote{$x$: Input tensor \\ $\theta$: Model parameters}
\section{无分类器引导 (CFG - Classifier-Free Guidance)}
Stable Diffusion 系列模型“听懂”人类 Prompt 的绝对核心数学技巧。

\begin{rosetta}{CFG 向量外插}
    在每个去噪步 $t$,模型同时预测有条件(Prompt $c$)和无条件(空文本 $\emptyset$)的噪声。
    \begin{equation}
        \hat{\epsilon}_\theta(x_t, c) = \epsilon_\theta(x_t, \emptyset) + w \cdot \left( \epsilon_\theta(x_t, c) - \epsilon_\theta(x_t, \emptyset) \right)
    \end{equation}
    其中 $w$ 即为 \texttt{guidance\_scale}。当 $w > 1$ 时,它在几何空间中强行放大了向条件 $c$ 靠近的方向向量,从而生成高度匹配 Prompt 但可能略微降低多样性的图像。
\end{rosetta}

\section{流匹配与整流 (Flow Matching \& Rectified Flows)}
Sora 和 Stable Diffusion 3 抛弃了传统的 DDPM 马尔可夫链,转向基于连续常微分方程 (ODE) 的直接轨迹预测。

\begin{rosetta}{Rectified Flows (ODE)}
    建立一条从纯噪声 $X_0 \sim \mathcal{N}(0, I)$ 到真实数据 $X_1$ 的直线轨迹:
    \begin{equation}
        X_t = t X_1 + (1 - t) X_0, \quad t \in [0, 1]
    \end{equation}
    神经网络 $v_\theta(X_t, t)$ 的目标是直接预测这条直线的导数(即流场):
    \begin{equation}
        \mathcal{L} = \mathbb{E}_{t, X_0, X_1} \left[ \| v_\theta(X_t, t) - (X_1 - X_0) \|_2^2 \right]
    \end{equation}
    这比扩散模型试图预测加入的随机噪声在数学上更加优美,且在采样时能以极少的步数求得 ODE 积分。
\end{rosetta}
