\chapter{概率论基础 (Probability Foundations)}
\label{ch:probability}

概率论是深度学习处理不确定性、构建生成模型和设计损失函数的基石。所有的变分推断、扩散模型和信息论指标都建立在概率分布的基础之上。

\begin{notation}
\begin{itemize}
    \item $x$: Input tensor
    \item $\mu$: Mean
    \item $\mathbf{x}$: Input vector
\end{itemize}
\end{notation}

\mnote{$x$: Input tensor \\ $\mu$: Mean \\ $\mathbf{x}$: Input vector}
\section{常见概率分布 (Common Distributions)}

\subsection{正态分布 (Normal Distribution)}
正态分布(高斯分布)是自然界中最常见的连续概率分布,也是许多生成模型(如 VAE 和 Diffusion)的先验分布假设。

\begin{rosetta}{多元正态分布}
    \begin{equation}
        \mathcal{N}(\mathbf{x}|\boldsymbol{\mu}, \boldsymbol{\Sigma}) = \frac{1}{(2\pi)^{D/2}|\boldsymbol{\Sigma}|^{1/2}} \exp\left(-\frac{1}{2}(\mathbf{x}-\boldsymbol{\mu})^\top \boldsymbol{\Sigma}^{-1} (\mathbf{x}-\boldsymbol{\mu})\right)
    \end{equation}
    \tcblower
    \pyfunc{torch.distributions.MultivariateNormal(loc, covariance\_matrix)}
\end{rosetta}

\subsection{伯努利与多项式分布 (Bernoulli \& Categorical)}
离散分布常用于分类任务的标签建模。

\begin{rosetta}{分类分布 (Categorical Distribution)}
    \begin{equation}
        P(X=k) = p_k, \quad \sum_{k=1}^K p_k = 1
    \end{equation}
    \tcblower
    \pyfunc{torch.distributions.Categorical(probs=p)}
\end{rosetta}

\section{贝叶斯定理与变分推断 (Bayesian \& VI)}

贝叶斯定理是生成模型(如 VAE)处理潜在变量(Latent Variables)的核心理论。

\begin{rosetta}{贝叶斯定理 (Bayes' Theorem)}
    \begin{equation}
        P(Z|X) = \frac{P(X|Z)P(Z)}{P(X)}
    \end{equation}
    其中 $P(Z|X)$ 是后验,$P(X|Z)$ 是似然,$P(Z)$ 是先验。
    \tcblower
    在深度学习中,我们通常使用编码器 $q_\phi(Z|X)$ 来近似难以计算的真后验 $P(Z|X)$。
\end{rosetta}

\section{马尔可夫链 (Markov Chains)}
马尔可夫链描述了一个状态空间中经过从一个状态到另一个状态的转换的随机过程。该过程要求具备“无记忆”的性质(即马尔可夫性质)。

\begin{figure}[htbp]
    \centering
    \begin{tikzpicture}[
        node distance=2.5cm,
        state/.style={circle, draw=maintheme, thick, fill=maintheme!10, minimum size=1.2cm, align=center},
        arrow/.style={-stealth, thick, draw=black!70}
    ]
        \node[state] (S1) {$S_1$};
        \node[state, right of=S1, xshift=1.5cm] (S2) {$S_2$};
        \node[state, right of=S2, xshift=1.5cm] (S3) {$S_3$};

        \draw[arrow] (S1) edge[bend left] node[above] {$P_{12}$} (S2);
        \draw[arrow] (S2) edge[bend left] node[below] {$P_{21}$} (S1);
        \draw[arrow] (S2) edge[bend left] node[above] {$P_{23}$} (S3);
        \draw[arrow] (S3) edge[bend left] node[below] {$P_{32}$} (S2);
        
        \draw[arrow] (S1) edge[loop above] node[above] {$P_{11}$} (S1);
        \draw[arrow] (S2) edge[loop above] node[above] {$P_{22}$} (S2);
        \draw[arrow] (S3) edge[loop right] node[right] {$P_{33}$} (S3);
    \end{tikzpicture}
    \caption{简单的马尔可夫链状态转移图}
    \label{fig:markov_chain}
\end{figure}

马尔可夫性质的数学定义为:
\begin{equation}
    P(X_{n+1}=x | X_n=x_n, X_{n-1}=x_{n-1}, \dots, X_0=x_0) = P(X_{n+1}=x | X_n=x_n)
\end{equation}
