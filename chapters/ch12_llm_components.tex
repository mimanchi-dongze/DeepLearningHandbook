\chapter{大模型组件 (LLM Components)}
\label{ch:llm_components}

现代大语言模型(如 LLaMA、Mixtral)在 Transformer 基础架构上引入了更高效的算子,极大增强了模型的外推能力与推理速度。

\section{RMSNorm (Root Mean Square Normalization)}
抛弃了均值平移(Mean-centering),仅使用均方根缩放,从而节省了计算开销而性能基本无损。

\begin{rosetta}{RMSNorm}
    \begin{equation}
        \text{RMSNorm}(x) = \frac{x}{\text{RMS}(x)} \odot \gamma \quad \text{其中 } \text{RMS}(x) = \sqrt{\frac{1}{d} \sum_{i=1}^{d} x_i^2 + \epsilon}
    \end{equation}
    这里 $\gamma$ 是可学习的缩放参数,$\epsilon$ 为防止除零的微小常数。
    \tcblower
    \pyfunc{x / torch.sqrt(torch.mean(x**2) + eps) * weight}
\end{rosetta}

\section{SwiGLU (Swish Gated Linear Unit)}
LLaMA 中的默认激活机制,采用 Swish 变体控制信息流。

\begin{rosetta}{SwiGLU}
    \begin{equation}
        \text{SwiGLU}(x, W, V) = \text{Swish}(xW) \odot xV
    \end{equation}
    \textbf{其中 Swish 函数定义为}:
    \begin{equation}
        \text{Swish}_{\beta}(z) = z \cdot \sigma(\beta z) = \frac{z}{1 + e^{-\beta z}}
    \end{equation}
    当 $\beta=1$ 时,即为 SiLU 函数。
\end{rosetta}

\section{RoPE (Rotary Position Embedding)}
旋转位置编码通过将绝对位置注入为旋转矩阵,自然推导出了相对位置衰减特性。

\begin{rosetta}{RoPE (旋转位置编码)}
    对于位置 $m$ 的特征向量 $\mathbf{x} = [x_1, x_2, \dots, x_{d}]$,将其两两分组应用旋转矩阵:
    \begin{equation}
        f(\mathbf{x}, m) = \mathbf{x} R_{\Theta, m}^d
    \end{equation}
    \textbf{旋转矩阵 $R_{\Theta, m}^d$}:
    \begin{equation}
        \begin{bmatrix}
            \cos m\theta_1 & -\sin m\theta_1 & 0 & 0 & \dots \\
            \sin m\theta_1 & \cos m\theta_1 & 0 & 0 & \dots \\
            0 & 0 & \cos m\theta_2 & -\sin m\theta_2 & \dots \\
            0 & 0 & \sin m\theta_2 & \cos m\theta_2 & \dots \\
            \vdots & \vdots & \vdots & \vdots & \ddots
        \end{bmatrix}
    \end{equation}
    其中基频 $\theta_i = 10000^{-2(i-1)/d}$。
\end{rosetta}

\section{MoE (Mixture of Experts)}
通过路由机制(Routing)让每个 Token 仅激活部分参数(如 Top-2),实现参数量扩大而不增加前向推理时间。

\begin{rosetta}{Top-K 路由数学}
    \textbf{路由权重}: 对于输入 $x$,专家 $E_i$ 的权重为:
    \begin{equation}
        G(x)_i = \begin{cases}
            \frac{e^{x \cdot W_{r, i}}}{\sum_{j \in T} e^{x \cdot W_{r, j}}}, & \text{if } i \in T \\
            0, & \text{otherwise}
        \end{cases}
    \end{equation}
    其中 $T$ 是根据 $x \cdot W_r$ 选出的 Top-K 索引集合。
    
    \textbf{最终输出}:
    \begin{equation}
        y = \sum_{i=1}^{N} G(x)_i E_i(x)
    \end{equation}
\end{rosetta}
