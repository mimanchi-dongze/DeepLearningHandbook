\chapter{神经网络层}
\label{ch:layers}

神经网络层是参数的学习单元,它将基础算子封装成带有可学习权重的模块。

\marginpar{
    \centering
    \begin{tikzpicture}[scale=0.75, transform shape]
        % Input nodes
        \foreach \i in {1,2,3} {
            \node[neuron] (I\i) at (0,-\i*1.2) {};
        }
        % Hidden nodes
        \foreach \j in {1,2} {
            \node[active_neuron] (H\j) at (2,-0.6-\j*1.2) {};
        }
        % Connections
        \foreach \i in {1,2,3} {
            \foreach \j in {1,2} {
                \draw[conn] (I\i) -- (H\j);
            }
        }
        \node[layer_label] at (0,-4.5) {Input Layer};
        \node[layer_label] at (2,-4.5) {Linear Layer};
    \end{tikzpicture}
    \captionof{figure}{Linear Layer (Fully Connected)}
}

\section{线性变换与卷积}

\subsection{线性层 (Linear / Fully Connected)}
执行仿射变换,是多层感知机(MLP)的核心。

\begin{rosetta}{线性层}
    \begin{equation}
        \mathbf{y} = \mathbf{x} \mathbf{W}^T + \mathbf{b}
    \end{equation}
    其中 $\mathbf{W} \in \mathbb{R}^{out \times in}$ 是权重矩阵,$\mathbf{b} \in \mathbb{R}^{out}$ 是偏置向量。
    \tcblower
    \pyfunc{nn.Linear(in\_features, out\_features)}
\end{rosetta}

\subsection{2D 卷积层 (Convolutional Layer)}
卷积层通过局部连接和权值共享提取空间特征。注意:PyTorch 实际执行的是互相关运算。

\begin{rosetta}{2D 卷积}
    \textbf{输出尺寸计算公式}:
    \begin{equation}
        H_{out} = \left\lfloor \frac{H_{in} + 2 \times \text{padding}[0] - \text{dilation}[0] \times (\text{kernel\_size}[0] - 1) - 1}{\text{stride}[0]} + 1 \right\rfloor
    \end{equation}
    \textbf{互相关运算定义}:
    \begin{equation}
        \text{out}(N_i, C_{out_j}) = \text{bias}(C_{out_j}) + \sum_{k=0}^{C_{in}-1} \text{weight}(C_{out_j}, k) \star \text{input}(N_i, k)
    \end{equation}
    其中 $\star$ 在此表示离散互相关:$(f \star g)(n) = \sum_{m} f(m)g(n+m)$。
    \tcblower
    \pyfunc{nn.Conv2d(in\_channels, out\_channels, kernel\_size, stride, padding)}
\end{rosetta}

\subsection{池化层 (Pooling Layers)}
用于下采样,减少计算量并增强平移不变性。

\begin{rosetta}{最大池化 (Max Pooling)}
    \begin{equation}
        \text{out}(C, h, w) = \max_{m, n} \text{input}(C, h \times S + m, w \times S + n)
    \end{equation}
    其中 $m, n$ 在卷积核大小范围内滑动。
    \tcblower
    \pyfunc{nn.MaxPool2d(kernel\_size)}
\end{rosetta}
