\chapter{对齐与极端量化}
\label{ch:alignment_quantization}

当大模型具备涌现能力后,下一步是让其输出符合人类偏好(对齐),并能在消费级硬件上运行(量化)。

\begin{notation}
\begin{itemize}
    \item $y$: Output label
    \item $\theta$: Model parameters
    \item $\sigma$: Activation function
\end{itemize}
\end{notation}

\mnote{$y$: Output label \\ $\theta$: Model parameters \\ $\sigma$: Activation function}
\section{直接偏好优化 (DPO)}
彻底绕过了传统 RLHF 中复杂的强化学习 (PPO) 训练流程,直接在偏好数据上优化语言模型本身。

\begin{rosetta}{DPO 目标函数}
    设 $\pi_\theta$ 为正在训练的模型,$\pi_{ref}$ 为参考模型,$y_w$ 为人类偏好的回答,$y_l$ 为被拒绝的回答。
    \begin{equation}
        \mathcal{L}_{DPO} = -\mathbb{E}_{(x, y_w, y_l)} \left[ \log \sigma \left( \beta \log \frac{\pi_\theta(y_w|x)}{\pi_{ref}(y_w|x)} - \beta \log \frac{\pi_\theta(y_l|x)}{\pi_{ref}(y_l|x)} \right) \right]
    \end{equation}
    其中 $\beta$ 是控制偏离参考模型程度的温度系数,$\sigma$ 为 Sigmoid 函数。该公式隐式定义了奖励。
\end{rosetta}

\section{QLoRA 与 NF4 量化}
为了在单张显卡上微调大模型,QLoRA 引入了一种信息论上最优的新数据类型:4-bit NormalFloat (NF4)。

\begin{rosetta}{分位数归一化与双重量化}
    \textbf{NF4 的数学基础}:
    假设神经网络的权重服从均值为 0 的正态分布 $\mathcal{N}(0, \sigma^2)$。NF4 预先计算了该分布的 16 个等概率分位数作为量化点.
    \begin{equation}
        q_i = \frac{1}{2} \left( Q\left(\frac{i}{16}\right) + Q\left(\frac{i+1}{16}\right) \right)
    \end{equation}
    \textbf{双重量化 (Double Quantization)}: 
    为了存储量化所需的缩放常数 $c$,对这些常数进行二次 8-bit 量化,使得平均每参数的内存开销再降低 $0.37$ bits。
\end{rosetta}
