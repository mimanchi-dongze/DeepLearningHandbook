\chapter{距离与相似度度量}
\label{ch:distance}

度量空间是机器学习的基础。在优化模型时,我们需要数学工具来衡量两个向量或分布的相似程度。

\marginpar{
    \centering
    \begin{tikzpicture}[scale=0.8, transform shape]
        \draw[<->, thick, gray] (0,2) node[above]{$y$} -- (0,0) -- (2.5,0) node[right]{$x$};
        \coordinate (A) at (0.5,0.5);
        \coordinate (B) at (2,1.5);
        \draw[arrow_blue] (0,0) -- (A) node[below right, font=\sffamily\scriptsize]{$\mathbf{u}$};
        \draw[arrow_blue] (0,0) -- (B) node[above left, font=\sffamily\scriptsize]{$\mathbf{v}$};
        \draw[dashed, brandorange, thick] (A) -- (B) node[midway, above left, font=\sffamily\scriptsize]{$L_2$};
        \draw[dashed, brandgreen, thick] (A) -| (B) node[near end, right, font=\sffamily\scriptsize]{$L_1$};
    \end{tikzpicture}
    \captionof{figure}{Distance Metrics}
}

\begin{notation}
\begin{itemize}
    \item $x$: Input tensor
    \item $y$: Output label
    \item $L$: Loss function
\end{itemize}
\end{notation}

\mnote{$x$: Input tensor \\ $y$: Output label \\ $L$: Loss function}
\section{欧氏距离与均方误差 (MSE)}
在欧几里得空间中,两点之间的直线距离。

\begin{rosetta}{MSE / L2 Loss}
    \begin{equation}
        \text{MSE}(\mathbf{y}, \mathbf{\hat{y}}) = \frac{1}{N} \sum_{i=1}^{N} (y_i - \hat{y}_i)^2 = \frac{1}{N} \| \mathbf{y} - \mathbf{\hat{y}} \|_2^2
    \end{equation}
    从概率学角度,MSE 对应于目标变量服从高斯分布时的最大似然估计。
    \tcblower
    \pyfunc{nn.MSELoss()}
\end{rosetta}

\section{曼哈顿距离与 L1 Loss}
计算绝对误差,对异常值(Outliers)比 MSE 更具鲁棒性。

\begin{rosetta}{L1 Loss / MAE}
    \begin{equation}
        \text{MAE}(\mathbf{y}, \mathbf{\hat{y}}) = \frac{1}{N} \sum_{i=1}^{N} |y_i - \hat{y}_i| = \frac{1}{N} \| \mathbf{y} - \mathbf{\hat{y}} \|_1
    \end{equation}
    对应于目标变量服从拉普拉斯分布时的最大似然估计。
    \tcblower
    \pyfunc{nn.L1Loss()}
\end{rosetta}

\section{余弦相似度 (Cosine Similarity)}
用于衡量两个向量方向的差异,不考虑向量的绝对大小。常用于文本嵌入(Embeddings)匹配。

\begin{rosetta}{Cosine Similarity}
    \begin{equation}
        \text{CosSim}(\mathbf{u}, \mathbf{v}) = \frac{\mathbf{u} \cdot \mathbf{v}}{\|\mathbf{u}\|_2 \|\mathbf{v}\|_2} = \frac{\sum u_i v_i}{\sqrt{\sum u_i^2} \sqrt{\sum v_i^2}}
    \end{equation}
    余弦距离定义为 $1 - \text{CosSim}$。
    \tcblower
    \pyfunc{nn.CosineSimilarity(dim)}
\end{rosetta}
