\chapter{循环与序列层}
\label{ch:layers_seq}

针对序列数据,神经网络需要具备“记忆”能力,通过隐藏状态在时间步之间传递信息。

\marginpar{
    \centering
    \begin{tikzpicture}[scale=0.7, transform shape]
        \node[block] (RNN) {RNN Cell};
        \node[data_node, below=0.8cm of RNN] (X) {$x_t$};
        \node[data_node, above=0.8cm of RNN] (Y) {$h_t$};
        \node[data_node, left=0.8cm of RNN] (Hprev) {$h_{t-1}$};
        \node[data_node, right=0.8cm of RNN] (Hnext) {$h_{t}$};
        
        \draw[conn] (X) -- (RNN);
        \draw[arrow_blue] (RNN) -- (Y);
        \draw[conn] (Hprev) -- (RNN);
        \draw[arrow_blue] (RNN) -- (Hnext);
    \end{tikzpicture}
    \captionof{figure}{RNN Time Step}
}

\section{RNN (Recurrent Neural Network)}
最基础的循环单元,由于梯度消失问题,难以处理长序列。



\begin{rosetta}{基础 RNN}
    \begin{equation}
        h_t = \tanh(W_{ih} x_t + b_{ih} + W_{hh} h_{t-1} + b_{hh})
    \end{equation}
    \tcblower
    \pyfunc{nn.RNN(input\_size, hidden\_size)}
\end{rosetta}

\section{LSTM (Long Short-Term Memory)}
通过引入“细胞状态”(Cell State)和门控机制,有效解决了长距离依赖问题。

\begin{rosetta}{LSTM 门控逻辑}
    \textbf{更新过程}:
    \begin{align}
        i_t &= \sigma(W_{ii} x_t + b_{ii} + W_{hi} h_{t-1} + b_{hi}) \quad \text{(输入门)} \\
        f_t &= \sigma(W_{if} x_t + b_{if} + W_{hf} h_{t-1} + b_{hf}) \quad \text{(遗忘门)} \\
        g_t &= \tanh(W_{ig} x_t + b_{ig} + W_{hg} h_{t-1} + b_{hg}) \quad \text{(细胞候选)} \\
        o_t &= \sigma(W_{io} x_t + b_{io} + W_{ho} h_{t-1} + b_{ho}) \quad \text{(输出门)} \\
        c_t &= f_t \odot c_{t-1} + i_t \odot g_t \quad \text{(细胞状态更新)} \\
        h_t &= o_t \odot \tanh(c_t) \quad \text{(隐藏状态更新)}
    \end{align}
    \tcblower
    \pyfunc{nn.LSTM(input\_size, hidden\_size)}
\end{rosetta}

\section{GRU (Gated Recurrent Unit)}
LSTM 的简化版本,合并了门控,计算效率更高。

\begin{rosetta}{GRU 逻辑}
    \begin{align}
        r_t &= \sigma(W_{ir} x_t + b_{ir} + W_{hr} h_{t-1} + b_{hr}) \quad \text{(重置门)} \\
        z_t &= \sigma(W_{iz} x_t + b_{iz} + W_{hz} h_{t-1} + b_{hz}) \quad \text{(更新门)} \\
        n_t &= \tanh(W_{in} x_t + b_{in} + r_t \odot (W_{hn} h_{t-1} + b_{hn})) \\
        h_t &= (1 - z_t) \odot n_t + z_t \odot h_{t-1}
    \end{align}
    \tcblower
    \pyfunc{nn.GRU(input\_size, hidden\_size)}
\end{rosetta}
