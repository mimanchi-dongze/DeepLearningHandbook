\chapter{高级线性代数算子}
\label{ch:linalg_functions}

线性代数算子是深度学习底层的动力引擎。除了基础的矩阵乘法外,诸如奇异值分解(SVD)、特征值分解和爱因斯坦求和约定(Einsum)在现代模型压缩、稳定训练和高性能计算中扮演着至关重要的角色。

\section{矩阵分解 (Matrix Decomposition)}

\subsection{奇异值分解 (SVD)}
SVD 将任何矩阵分解为旋转、缩放和旋转三个过程。在深度学习中,常用于 LoRA 的权重初始化或模型压缩。

\begin{rosetta}{奇异值分解 (SVD)}
    \begin{equation}
        A = U \Sigma V^T
    \end{equation}
    其中:
    \begin{itemize}
        \item $U, V$ 是正交矩阵。
        \item $\Sigma$ 是对角矩阵,包含奇异值。
    \end{itemize}
    \tcblower
    \pyfunc{U, S, Vh = torch.linalg.svd(A)}
\end{rosetta}

\subsection{QR 分解}
QR 分解将矩阵分解为一个正交矩阵 $Q$ 和一个上三角矩阵 $R$。常用于数值稳定性要求极高的线性方程组求解。

\begin{rosetta}{QR 分解}
    \begin{equation}
        A = QR
    \end{equation}
    其中 $Q^T Q = I$,$R$ 为上三角。
    \tcblower
    \pyfunc{Q, R = torch.linalg.qr(A)}
\end{rosetta}

\section{矩阵属性与度量}

\subsection{行列式与逆}
虽然在反向传播中较少直接求逆,但在概率图模型或雅可比行列式计算中不可或缺。

\begin{rosetta}{行列式与逆矩阵}
    \begin{equation}
        \det(A) \cdot A^{-1} = \text{adj}(A)
    \end{equation}
    \tcblower
    \pyfunc{torch.linalg.det(A)} \\
    \pyfunc{torch.linalg.inv(A)}
\end{rosetta}

\subsection{矩阵范数 (Matrix Norm)}
用于度量矩阵的“大小”,在谱归一化(Spectral Normalization)中特指算子范数。

\begin{rosetta}{谱范数 (Spectral Norm)}
    \begin{equation}
        \|A\|_2 = \sigma_{\max}(A)
    \end{equation}
    即矩阵最大奇异值。
    \tcblower
    \pyfunc{torch.linalg.matrix\_norm(A, ord=2)}
\end{rosetta}

\section{高级张量约定}

\subsection{爱因斯坦求和约定 (Einsum)}
Einsum 是处理多维张量收缩的“瑞士军刀”,能以极简语法实现转置、乘法、迹和外积。

\begin{rosetta}{Einstein Summation}
    \begin{equation}
        C_{ij} = \sum_{k} A_{ik} B_{kj} \implies \texttt{ik,kj->ij}
    \end{equation}
    \tcblower
    \pyfunc{torch.einsum('ik,kj->ij', A, B)}
\end{rosetta}

\subsection{批量矩阵乘法 (BMM)}
在处理 Batch 数据(如 Attention 权重)时,高效处理前两个维度外的矩阵乘法。

\begin{rosetta}{Batch MatMul}
    \begin{equation}
        C_{b,i,j} = \sum_{k} A_{b,i,k} B_{b,k,j}
    \end{equation}
    \tcblower
    \pyfunc{torch.bmm(A, B)}
\end{rosetta}

\begin{center}
\begin{tikzpicture}[node distance=1.8cm, font=\sffamily\small]
    % SVD 示意图
    \node[data_node, fill=brandblue!10, minimum height=2cm] (A) {$A$};
    \node[right=1cm of A] (eq) {$=$};
    \node[data_node, right=1cm of eq, fill=brandorange!10, minimum height=2cm] (U) {$U$};
    \node[data_node, right=0.2cm of U, fill=brandgreen!10, minimum height=1cm, minimum width=1cm] (S) {$\Sigma$};
    \node[data_node, right=0.2cm of S, fill=brandblue!5, minimum width=2cm, minimum height=2cm] (V) {$V^T$};
    
    \node[layer_label, below=0.2cm of A] {Matrix};
    \node[layer_label, below=0.2cm of U] {Orthogonal};
    \node[layer_label, below=0.7cm of S] {Singular};
    \node[layer_label, below=0.2cm of V] {Orthogonal};
\end{tikzpicture}
\end{center}
