\chapter{生成模型数学基础}
\label{ch:generative}

生成模型旨在学习数据的概率分布 $p(x)$,从而生成新的样本。

\section{VAE (Variational Autoencoder)}
VAE 通过最大化变分下界(ELBO)来训练。

\subsection{重参数化技巧 (Reparameterization Trick)}
为了让随机采样过程可导,我们将随机性转移到外部噪声。这使得梯度能够流过分布参数 $\mu$ 和 $\sigma$。

\begin{center}
\begin{tikzpicture}[node distance=1.5cm, font=\sffamily\small]
    % 标准正态噪声
    \node[active_neuron] (eps) {$\epsilon$};
    \node[layer_label, above=0.2cm of eps] {$\mathcal{N}(0, I)$};
    
    % 分布参数
    \node[data_node, left=1.5cm of eps] (mu) {$\mu$};
    \node[data_node, right=1.5cm of eps] (sigma) {$\sigma$};
    
    % 变换操作
    \node[op_node, below=1cm of eps] (mul) {$\times$};
    \node[op_node, below=2cm of eps] (add) {$+$};
    
    \draw[conn] (sigma) |- (mul);
    \draw[conn] (eps) -- (mul);
    \draw[conn] (mul) -- (add);
    \draw[conn] (mu) |- (add);
    
    % 采样结果
    \node[data_node, below=0.8cm of add] (z) {Latent $z$};
    \draw[conn] (add) -- (z);
    
    % 梯度流向标注
    \path[->, >=Stealth, brandorange, dashed, thick] (z.north) edge[bend left=30] node[left] {Gradient Flow} (mu.south);
    \path[->, >=Stealth, brandorange, dashed, thick] (z.north) edge[bend right=30] node[right] {Gradient Flow} (sigma.south);
\end{tikzpicture}
\end{center}

\begin{rosetta}{重参数化}
    \begin{equation}
        \mathbf{z} = \mu + \sigma \odot \epsilon, \quad \epsilon \sim \mathcal{N}(0, I)
    \end{equation}
    \tcblower
    \texttt{z = mu + sigma * torch.randn\_like(sigma)}
\end{rosetta}

\subsection{ELBO 损失函数}
VAE 的目标函数由两部分组成:重建误差和正则化项。

\begin{rosetta}{VAE 损失定义}
    \begin{equation}
        \mathcal{L}_{VAE} = \mathbb{E}_{q(z|x)}[\log p(x|z)] - D_{KL}(q(z|x) \| p(z))
    \end{equation}
    其中第一项是重建似然(通常用 MSE 或 BCE),第二项是隐藏变量分布与先验分布(通常为标准正态)的 KL 散度。
\end{rosetta}

\section{Diffusion Model (扩散模型) 简介}
扩散模型通过逐渐向数据添加噪声(正向过程)和学习去噪(逆向过程)来生成图像。

\begin{center}
\begin{tikzpicture}[node distance=2.5cm, font=\sffamily\small]
    % 状态节点
    \node[data_node, fill=brandblue!10] (x0) {$x_0$};
    \node[data_node, right of=x0] (xt1) {$x_{t-1}$};
    \node[data_node, right of=xt1] (xt) {$x_t$};
    \node[data_node, right of=xt, fill=brandorange!10] (xT) {$x_T$};
    
    % 前向过程 (加噪)
    \draw[data_flow, bend left=25] (x0) to node[above] {$q(x_{t-1}|x_0)$} (xt1);
    \draw[data_flow, bend left=25] (xt1) to node[above] {$q(x_t|x_{t-1})$} (xt);
    \draw[data_flow, bend left=25] (xt) to node[above] {$\dots$} (xT);
    
    % 逆向过程 (去噪)
    \draw[conn, draw=brandorange, bend left=25] (xT) to node[below] {$p_\theta(x_{t-1}|x_t)$} (xt);
    \draw[conn, draw=brandorange, bend left=25] (xt) to node[below] {$\dots$} (xt1);
    \draw[conn, draw=brandorange, bend left=25] (xt1) to node[below] {$p_\theta(x_0|x_1)$} (x0);
    
    % 标签
    \node[layer_label, below=0.3cm of x0] {DATA};
    \node[layer_label, below=0.3cm of xT] {NOISE};
\end{tikzpicture}
\end{center}

\begin{rosetta}{DDPM 正向扩散}
    \begin{equation}
        q(x_t | x_{t-1}) = \mathcal{N}(x_t; \sqrt{1-\beta_t}x_{t-1}, \beta_t I)
    \end{equation}
    通过重参数化,我们可以直接从 $x_0$ 计算 $x_t$。
\end{rosetta}
