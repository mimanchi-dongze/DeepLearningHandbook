\chapter{参数高效微调 (PEFT \& LoRA)}
\label{ch:peft_lora}

当模型规模增长至百亿甚至千亿参数时,全参微调(Full Fine-Tuning)的显存和计算成本难以承受。低秩自适应(Low-Rank Adaptation, LoRA)通过极少的额外参数,保留了原始知识并适应新任务。

\begin{notation}
\begin{itemize}
    \item $x$: Input tensor
    \item $y$: Output label
    \item $W$: Weight matrix
    \item $\sigma$: Activation function
    \item $\mathbf{x}$: Input vector
    \item $h$: Hidden state
\end{itemize}
\end{notation}

\mnote{$x$: Input tensor \\ $y$: Output label \\ $W$: Weight matrix \\ $\sigma$: Activation function \\ $\mathbf{x}$: Input vector}
\mnote{$h$: Hidden state}
\section{矩阵低秩分解}

LoRA 的核心思想是冻结预训练模型的权重矩阵,并通过旁路的两个低秩矩阵之积来表示权重的增量($\Delta W$)。

\begin{center}
\begin{tikzpicture}[node distance=1.5cm, font=\sffamily\small]
    % 输入
    \node (x) at (0,0) {$x \in \mathbb{R}^k$};
    
    % 主路径:冻结权重
    \node[block, above=1cm of x, minimum height=3cm, fill=brandblue!10] (w0) {Frozen Pretrained \\ $W_0 \in \mathbb{R}^{d \times k}$};
    \draw[conn] (x) -- (w0);
    
    % 旁路:LoRA 路径
    \node[data_node, right=2cm of w0] (matA) {$A \in \mathbb{R}^{r \times k}$};
    \node[data_node, above=1.5cm of matA] (matB) {$B \in \mathbb{R}^{d \times r}$};
    
    \draw[conn] (x.east) -- ++(1,0) |- (matA.west);
    \draw[conn] (matA) -- node[right, math_text] {rank $r \ll d, k$} (matB);
    
    % 合并
    \node[op_node, above=4.5cm of x] (sum) {$+$};
    \draw[conn] (w0) -- (sum);
    \draw[conn] (matB) |- (sum);
    
    % 输出
    \node[above=0.8cm of sum] (out) {$h = W_0 x + BAx$};
    \draw[conn] (sum) -- (out);

    % 标注
    \begin{scope}[on background layer]
        \node[container, fit=(matA) (matB)] (lorabox) {};
        \node[anchor=south west, brandorange] at (lorabox.north west) {Trainable LoRA};
    \end{scope}
\end{tikzpicture}
\end{center}

\begin{rosetta}{LoRA 更新公式}
    设预训练权重矩阵为 $W_0 \in \mathbb{R}^{d \times k}$,冻结不变。
    更新量 $\Delta W$ 通过两个低秩矩阵 $B \in \mathbb{R}^{d \times r}$ 和 $A \in \mathbb{R}^{r \times k}$ 乘积得到:
    \begin{equation}
        W = W_0 + \Delta W = W_0 + B A
    \end{equation}
    其中秩 $r \ll \min(d, k)$,例如 $r=8$ 或 $r=64$。
\end{rosetta}

\section{前向传播与初始化}

\subsection{前向计算逻辑}
对于输入 $\mathbf{x} \in \mathbb{R}^k$,全连接层的原本输出为 $h = W_0 \mathbf{x}$。引入 LoRA 之后,前向传播变为两个并行的矩阵乘法最后相加。

\begin{rosetta}{LoRA 前向传播}
    \begin{equation}
        h = W_0 \mathbf{x} + \frac{\alpha}{r} B A \mathbf{x}
    \end{equation}
    其中 $\alpha$ 是一个缩放超参数,在初始阶段控制新加入参数对最终结果的影响。当秩 $r$ 改变时,可以通过 $\frac{\alpha}{r}$ 保持方差稳定。
    \tcblower
    \pyfunc{y = F.linear(x, W0) + F.linear(x, B @ A) * (alpha / r)}
\end{rosetta}

\subsection{零初始化技巧}
为了保证在训练刚开始时(即第一个 Step 前),引入的 LoRA 模块不会破坏预训练模型的行为,矩阵 $A$ 和 $B$ 的初始化至关重要。

\begin{rosetta}{LoRA 初始化}
    \textbf{矩阵 A}: 使用标准正态分布初始化(Gaussian / Kaiming)。
    \begin{equation}
        A \sim \mathcal{N}(0, \sigma^2)
    \end{equation}
    \textbf{矩阵 B}: 必须使用全零矩阵初始化。
    \begin{equation}
        B = \mathbf{0} \implies B A = \mathbf{0}
    \end{equation}
    这样,初始状态下 $\Delta W = 0$,保证了 $h = W_0 \mathbf{x}$ 与原始模型输出完全一致。
\end{rosetta}

\section{推理合并 (Merging Weights)}

在推理阶段,由于矩阵乘法的分配律,我们可以直接将 $BA$ 加回到 $W_0$ 中。因此,部署带有 LoRA 的大模型,其**推理延迟与基座模型完全相同**(Zero Latency Cost)。

\begin{rosetta}{LoRA 部署公式}
    \begin{equation}
        W_{deploy} = W_0 + B A
    \end{equation}
    在 PyTorch 中只需执行 \pyfunc{W0.add\_(B @ A)} 即可卸载独立参数。
\end{rosetta}
