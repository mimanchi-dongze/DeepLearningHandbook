\chapter{张量基础运算}
\label{ch:basics}

在深度学习中,张量(Tensor)是承载数据的基本容器。所有的复杂层最终都会分解为对这些多维数组的基础算子。

\marginpar{
    \centering
    \begin{tikzpicture}[scale=0.8, transform shape]
        \node[block] (A) {Tensor A};
        \node[block_orange, below=0.6cm of A] (OP) {Operation};
        \node[block_green, below=0.6cm of OP] (C) {Tensor C};
        \draw[conn] (A) -- (OP);
        \draw[arrow_blue] (OP) -- (C);
    \end{tikzpicture}
    \captionof{figure}{Tensor Operation Flow}
}

\section{逐元素运算 (Element-wise Operations)}

\subsection{Hadamard 乘积}
逐元素乘法是指两个形状相同的张量,对应位置的元素相乘。

\begin{rosetta}{逐元素乘法}
    \begin{equation}
        C = A \odot B \quad \text{其中 } C_{i,j} = A_{i,j} \times B_{i,j}
    \end{equation}
    \tcblower
    \pyfunc{torch.mul(A, B)} 或 \pyfunc{A * B}
\end{rosetta}

\subsection{幂运算}
对张量中的每一个分量求 $n$ 次幂。

\begin{rosetta}{幂运算}
    \begin{equation}
        y = x^n
    \end{equation}
    \tcblower
    \pyfunc{torch.pow(x, n)}
\end{rosetta}

\section{线性代数 (Linear Algebra)}

\subsection{矩阵乘法}
这是神经网络中线性变换的核心,常用于全连接层和注意力机制。

\begin{rosetta}{矩阵乘法}
    \begin{equation}
        C = AB \implies C_{i,j} = \sum_{k=1}^{n} A_{i,k}B_{k,j}
    \end{equation}
    \tcblower
    \pyfunc{torch.matmul(A, B)} 或 \pyfunc{A @ B}
\end{rosetta}

\subsection{转置}
交换张量的维度,在矩阵中通常指行与列的互换。

\begin{rosetta}{转置}
    \begin{equation}
        A_{i,j}^T = A_{j,i}
    \end{equation}
    \tcblower
    \pyfunc{tensor.T} 或 \pyfunc{tensor.t()}
\end{rosetta}

\section{统计规约 (Reduction)}

\subsection{L2 范数}
计算向量或矩阵的模长。

\begin{rosetta}{L2 范数}
    \begin{equation}
        \|\mathbf{x}\|_2 = \sqrt{\sum_{i=1}^{n} |x_i|^2}
    \end{equation}
    \tcblower
    \pyfunc{torch.norm(x, p=2)}
\end{rosetta}
