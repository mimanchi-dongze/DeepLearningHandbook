\chapter{随机过程 (Stochastic Processes)}
\label{ch:stochastic}

随机过程是处理时间序列、扩散模型前向与反向加噪的核心数学工具。我们将在此探讨布朗运动与随机微分方程(SDE)。

\section{随机微分方程 (Stochastic Differential Equations)}

\begin{variable}
\begin{itemize}
    \item $\mathbf{x} \in \mathbb{R}^d$: 系统状态(如图像张量的扁平化向量)。
    \item $\mathbf{f}(\mathbf{x}, t) \in \mathbb{R}^d$: 漂移项 (Drift),决定确定性的演化趋势。
    \item $g(t) \in \mathbb{R}$: 扩散项 (Diffusion),决定随机噪声的强度。
    \item $\mathbf{w} \in \mathbb{R}^d$: 标准维纳过程,其微分 $d\mathbf{w}$ 代表白噪声。
    \item $\epsilon \sim \mathcal{N}(0, 1)$: 标准正态分布采样,用于离散化模拟。
\end{itemize}
\end{variable}


在扩散模型(如 Score-based Generative Models)中,数据受到由 SDE 控制的演化:
\begin{equation}
    d\mathbf{x} = \mathbf{f}(\mathbf{x}, t)dt + g(t)d\mathbf{w}
\end{equation}

其中 $\mathbf{f}(\mathbf{x}, t)$ 是漂移系数(Drift),$g(t)$ 是扩散系数(Diffusion)。

\begin{rosetta}{欧拉-丸山方法 (Euler-Maruyama Method)}
    对于 SDE $dx_t = f(x_t, t)dt + g(t)dw_t$,其数值离散解为:
    \begin{equation}
        x_{t+\Delta t} = x_t + f(x_t, t)\Delta t + g(t)\sqrt{\Delta t} \cdot \epsilon, \quad \epsilon \sim \mathcal{N}(0, 1)
    \end{equation}
    \tcblower
    \textbf{PyTorch 实现}:
    \begin{verbatim}
# x: current state, t: time, dt: step
eps = torch.randn_like(x)
drift = f(x, t) * dt
diffusion = g(t) * torch.sqrt(dt) * eps
x_next = x + drift + diffusion
    \end{verbatim}
\end{rosetta}

\section{SDE 轨迹 (SDE Trajectories)}

随机微分方程的每一次采样都会产生一条不同的轨迹(Trajectory)。

\begin{figure}[htbp]
    \centering
    \begin{tikzpicture}
        % Axes
        \draw[->, thick] (0,0) -- (8,0) node[right] {$t$};
        \draw[->, thick] (0,-3) -- (0,3) node[above] {$X_t$};
        
        % Data distribution at t=0
        \draw[fill=maintheme, opacity=0.3] (0,1.5) ellipse (0.2 and 0.8);
        \draw[fill=maintheme, opacity=0.3] (0,-1.5) ellipse (0.2 and 0.8);
        \node[left] at (-0.3, 1.5) {$\text{Data}(t=0)$};
        
        % Trajectory 1
        \draw[thick, draw=blue!70, decorate, decoration={random steps,segment length=5pt,amplitude=2.5pt}] 
            (0,1.5) -- (1,1.2) -- (2,1.6) -- (3,0.5) -- (4,0.8) -- (5,0.1) -- (6,0.5) -- (7,0.2);
            
        % Trajectory 2
        \draw[thick, draw=red!70, decorate, decoration={random steps,segment length=5pt,amplitude=2.5pt}] 
            (0,-1.5) -- (1,-1.0) -- (2,-1.2) -- (3,-0.5) -- (4,0.1) -- (5,-0.2) -- (6,0.3) -- (7,-0.1);
            
        % Trajectory 3
        \draw[thick, draw=green!70!black, decorate, decoration={random steps,segment length=5pt,amplitude=2.5pt}] 
            (0,0.8) -- (1,1.5) -- (2,0.5) -- (3,0.8) -- (4,0.2) -- (5,0.6) -- (6,-0.2) -- (7,0.4);
            
        % Noise distribution at t=T
        \draw[fill=gray, opacity=0.3] (7,0) ellipse (0.4 and 2);
        \node[right] at (7.5, 0) {$\text{Noise}(t=T)$};
        
        \node[below] at (3.5, -2.5) {前向加噪过程 (Forward Process SDE)};
    \end{tikzpicture}
    \caption{数据分布通过随机微分方程演化为噪声分布的可能轨迹}
    \label{fig:sde_trajectory}
\end{figure}

轨迹的演化展示了扩散模型前向加噪过程的几何直觉,数据逐步从复杂分布演化向各向同性的高斯分布。
