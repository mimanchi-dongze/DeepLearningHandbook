\chapter{激活函数}
\label{ch:activations}

激活函数通过引入非线性变换,使得神经网络能够拟合极其复杂的函数分布。

\marginpar{
    \centering
    \begin{tikzpicture}[scale=0.8]
        \draw[->, gray] (-2,0) -- (2,0) node[right] {$x$};
        \draw[->, gray] (0,-1) -- (0,2) node[above] {$y$};
        \draw[thick, brandorange] (-1.5,0) -- (0,0) -- (1.5,1.5) node[right] {ReLU};
    \end{tikzpicture}
    \captionof{figure}{ReLU Function}
}

\begin{notation}
\begin{itemize}
    \item $x$: Input tensor
    \item $y$: Output label
    \item $\sigma$: Activation function
\end{itemize}
\end{notation}

\mnote{$x$: Input tensor \\ $y$: Output label \\ $\sigma$: Activation function}
\section{常用激活函数}

\subsection{ReLU (Rectified Linear Unit)}
ReLU 是深度学习中最常用的激活函数,因其计算简单且能有效缓解梯度消失问题。

\begin{rosetta}{ReLU 激活函数}
    \begin{equation}
        \text{ReLU}(x) = \max(0, x)
    \end{equation}
    \tcblower
    \pyfunc{torch.relu(x)} 或 \pyfunc{nn.ReLU()}
\end{rosetta}

\subsection{Sigmoid}
将输入映射到 $(0, 1)$ 区间,常用于二分类问题的概率预测。

\begin{rosetta}{Sigmoid 激活函数}
    \begin{equation}
        \sigma(x) = \frac{1}{1 + e^{-x}}
    \end{equation}
    \tcblower
    \pyfunc{torch.sigmoid(x)} 或 \pyfunc{nn.Sigmoid()}
\end{rosetta}

\subsection{Tanh (双曲正切)}
将输入映射到 $(-1, 1)$ 区间,输出均值为 0,在某些网络中比 Sigmoid 收敛更快。

\begin{rosetta}{Tanh 激活函数}
    \begin{equation}
        \tanh(x) = \frac{e^x - e^{-x}}{e^x + e^{-x}}
    \end{equation}
    \tcblower
    \pyfunc{torch.tanh(x)} 或 \pyfunc{nn.Tanh()}
\end{rosetta}

\subsection{Softmax}
常用于多分类任务的输出层,将一组向量转化为概率分布。

\begin{rosetta}{Softmax 函数}
    \begin{equation}
        \text{Softmax}(x_i) = \frac{e^{x_i}}{\sum_{j=1}^{C} e^{x_j}}
    \end{equation}
    \tcblower
    \pyfunc{torch.softmax(x, dim)}
\end{rosetta}
