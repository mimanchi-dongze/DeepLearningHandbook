\chapter{信息论与概率损失}
\label{ch:information}

损失函数用于衡量模型预测分布与真实分布之间的差异。在分类生成任务中,基于信息论(Information Theory)的度量是最基础的法则。

\begin{notation}
\begin{itemize}
    \item $x$: Input tensor
    \item $y$: Output label
\end{itemize}
\end{notation}

\mnote{$x$: Input tensor \\ $y$: Output label}
\section{香农熵与交叉熵 (Entropy \& Cross-Entropy)}

香农熵(Entropy)衡量了一个分布的不确定性。当我们使用模型预测的分布 $Q$ 来逼近真实分布 $P$ 时,需要的额外信息量就是交叉熵。

\begin{rosetta}{交叉熵损失}
    \begin{align}
        H(P, Q) &= -\sum_{x} P(x) \log Q(x) \\
        &= H(P) + D_{KL}(P \| Q)
    \end{align}
    其中 $H(P)$ 是真实分布的固有熵。在独热编码(One-Hot)下,真实类别概率 $P(x_i)=1$,固有熵为 0,交叉熵退化为:
    \begin{equation}
        \mathcal{L}_{CE} = -\log(\hat{y}_c)
    \end{equation}
    这正是 \pyfunc{nn.NLLLoss(LogSoftmax(x))} 的底层逻辑。
    \tcblower
    \pyfunc{nn.CrossEntropyLoss()}
\end{rosetta}

\section{KL 散度 (Kullback-Leibler Divergence)}
用于衡量两个概率分布(通常为预测与目标或先验)之间的“距离”(相对熵,不对称)。

\begin{rosetta}{KL 散度}
    \begin{equation}
        D_{KL}(P \| Q) = \sum_{x} P(x) \log\left(\frac{P(x)}{Q(x)}\right)
    \end{equation}
    在 VAE 中,KL 散度常被用于正则化潜变量(Latent Variable)分布使其逼近标准正态分布 $\mathcal{N}(0, I)$。
    \tcblower
    \pyfunc{nn.KLDivLoss(reduction='batchmean')}
\end{rosetta}

\section{二元交叉熵 (BCE)}
多标签分类或二分类的专用交叉熵。

\begin{rosetta}{BCE Loss}
    \begin{equation}
        \mathcal{L}_{BCE} = -\frac{1}{N} \sum_{i=1}^{N} \left[ y_i \log(\hat{y}_i) + (1 - y_i) \log(1 - \hat{y}_i) \right]
    \end{equation}
    这是伯努利分布下最大似然估计的直接结果。
    \tcblower
    \pyfunc{nn.BCELoss()} / \pyfunc{BCEWithLogitsLoss()}
\end{rosetta}
